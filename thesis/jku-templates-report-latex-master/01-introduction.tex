%%%%%%%%%%%%%%%%%%%%%%%%%%%%%%%%%%%%%%%%%%%%%%%%%%%%%%%%%%%%%%%%%%%%%%%%%%%%%%%%
%% 
\cleardoubleoddpage%
\chapter{Introduction}
\label{sec:introduction}
The Semantic Web represents an evolution of the current web, extending it from a web of documents to a web of data. It's central goal is to enable machines to understand and process information by encoding the semantics of data in standardized, machine-interpretable formats. Ontologies form the backbone of this infrastructure, providing formal and explicit specifications of conceptualizations that describe entities within a given domain and the relationships between them. Through their use, heterogeneous data sources can be linked, integrated, and semantically enriched, enabling knowledge inference and interoperability across systems \cite{bib:gruber-ontologies} \cite{bib:2006-sw-lee}.

To describe and exchange such knowledge, ontology representation languages and data models such as the Resource Description Framework (RDF), RDF Schema (RDFS), and the Web Ontology Language (OWL) have been established \footnote{\url{https://www.w3.org/RDF/}} \footnote{\url{https://www.w3.org/TR/rdf-schema/}} \footnote{\url{https://www.w3.org/OWL/}}. These form the technical foundation for defining classes, properties, and individuals as well as the logical axioms that capture semantic relationships. Alongside these representations, vocabularies—sets of shared, reusable properties and classes—enable the consistent annotation of both data and ontological resources. Well-known examples include Dublin Core Terms (DCTERMS), the Vocabulary of Interlinked Datasets (VoID), schema.org, and the Provenance Ontology (PROV-O) \footnote{\url{https://www.dublincore.org/specifications/dublin-core/dcmi-terms/}} \footnote{\url{https://www.w3.org/TR/void/}} \footnote{\url{https://schema.org/}} \footnote{\url{https://www.w3.org/TR/prov-o/}}.

Despite these advances, the annotation of ontologies themselves with metadata remains insufficiently standardized. Classic metadata such as creator, version, license, or domain coverage are essential for assessing the quality, provenance, and reusability of an ontology. Nevertheless, ontology designers often use different vocabularies or apply properties inconsistently, resulting in reduced interoperability and discoverability. Consequently, automated documentation and ontology discovery become difficult, as metadata cannot be reliably interpreted by tools or search services. This gap is particularly relevant for ontology documentation tools such as WIDOCO or pyLODE, which rely on embedded metadata to generate human-readable descriptions \footnote{\url{https://dgarijo.github.io/Widoco/}} \footnote{\url{https://pypi.org/project/pylode/}} . Although these tools facilitate documentation generation, they offer limited guidance on how to annotate metadata correctly and comprehensively \cite{bib:modPaper}. 

The absence of a unified standard for ontology metadata annotation motivates the central objective of this thesis: the development of a vocabulary that enables consistent, interoperable, and semantically precise annotation of ontologies. The aim is to capture all conceivable metadata that someone could annotate in the resulting vocabulary. Three main design principles are followed: the reuse of existing, well-established vocabularies wherever possible, the extension of these vocabularies to address ontology-specific metadata needs, and adherence to the FAIR data principles. The resulting vocabulary aims to improve ontology documentation, discovery, and reuse within the Semantic Web ecosystem.


The thesis is structured as follows: Chapter \ref{sec:Background} provides an overview of basic concepts of the semantic web, while Chapter \ref{sec:Design} presents a detailed analysis of existing metadata vocabularies and their applicability to ontology annotation, followed by the design and implementation of the proposed vocabulary. Finally, Chapter \ref{sec:conclusion} concludes the thesis by summarizing the results and outlining directions for future work, as well as provide an outlook at current topics such as knowledge graphs.
%% 
%%%%%%%%%%%%%%%%%%%%%%%%%%%%%%%%%%%%%%%%%%%%%%%%%%%%%%%%%%%%%%%%%%%%%%%%%%%%%%%%
